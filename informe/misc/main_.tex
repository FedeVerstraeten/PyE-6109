\documentclass{article}
\usepackage[utf8]{inputenc}

\usepackage{amsmath}
\usepackage{amssymb}
\usepackage{siunitx}

\begin{document}

\section{Análisis de los gráficos}

Es de interés obtener la función de distribución empírica y el histograma de $X_1$ o de $X_2$ para comparar los gráficos contra lo esperado si fueran variables aleatorias normales estándar independientes. Dado que $X_1$ y $X_2$ están correlacionadas, se podría esperar que las variables condicionadas $X_1|X_2$ o $X_2|X_1$ tengan una función de distribución y una función densidad distintas a las de una normal estándar. Sin embargo, como se pide en el enunciado, se obtuvieron las funciones de distribución e histograma de las variables $X_1$ y $X_2$ (notesé que no son las mismas variables aleatorias). Los resultados obtenidos para $X_2$ pueden verse en las figuras (REFERENCIAR FIGURAS Fde Y Hist).\\

Tal como se observa, las formas que tienen los gráficos tienden a coincidir con las de normales estándar. La función de distribución ``\emph{acumula de forma simétrica probabilidad}'' desde el comienzo de la simulación hasta el final, teniendo aproximadamente $0.5$ del total a un lado y a al otro desde su media que coincide aproximadamente con $0$.\\

La función histograma se parece bastante a una forma ``\emph{discretizada}'' de la campana de Gauss esperada. Aquí se puede observar que la media esta cercana al $0$ y que el desvío (distancia desde la media hasta cubrir la mayor parte de la densidad) tiende a $1$. Para realizar la función histograma se plantearon los intervalos propuestos por la cátedra en el enunciado del trabajo práctico. Estos intervalos son:\\

(AGREGAR INTERVALOS, CREO QUE ESTO YA LO TENIAMOS ESCRITO EN ALGUNA PARTE, PERO FALTA REEMPLAZAR a0 = -4.5 y a10 = 4.5 : TE PASO DIRECTAMENTE LAS FRECUENCIAS ABSOLUTAS DE CADA INTERVALO Y LOS VALORES DEL HISTOGRAMA, ES DECIR LA ALTURA DE CADA BARRA)\\

Puede observarse que si se usaran mayor cantidad de intervalos cerca del centro de la ``\emph{campana}'', se podría ver con más claridad en el gráfico la tendencia a una función de densidad de una $\mathcal{N}$($0$,$1$).

\end{document}