\documentclass{article}
\usepackage[utf8]{inputenc}

\usepackage{amsmath}
\usepackage{amssymb}
\usepackage{siunitx}

\begin{document}

\section{Ej. Pablo}

\subsection{Inciso 5}

	Como se puede observar en las figuras (REFERENCIAR FIGURAS NORMALES BIV),  las normales simuladas $(X_1,X_2)$ tienen una distribución conjunta cuyo soporte no es de forma rectangular. En este sentido, se puede descartar la hipótesis de que sean independientes. Además, 

\[X_1 , X_2 \textrm{ son independientes} \iff \rho = 0\]\\

	Esto se puede observar en la figura (REFERENCIAR FIG NORMAL BIV RHO 0), cuyo soporte es "más rectangular". Sin embargo, en las simulaciones pedidas el coeficiente de correlación $\rho$ es cercano a 1 en todos los casos, por lo que $X_1$ y $X_2$ no son independientes. Por lo tanto, se puede pensar que:

\[\mathbb{P}(X_1 \leq 1 \cap X_1 \leq 1) \neq \mathbb{P}(X_1 \leq 1) \cdot \mathbb{P}(X_2 \leq 1)\]

\[\implies \mathbb{P}(X_1 \leq 1 \cap X_1 \leq 1) \neq (\Phi(1))^2\]\\

	En este sentido, se implementó un ciclo que contara los casos favorables de la simulación (es decir, que cumplieran la condición de la intersección) y se los dividió por los casos totales (es decir, el tamaño de la simulación). Los resultados obtenidos son :\\

(SI PODÉS, PRESENTA LOS RESULTADOS EN UNA TABLA QUE TENGA PADRON USADO, RHO, CASOS FAVORABLES Y TOTALES, PROBABILIDAD CALCULADA Y EL $(\Phi(1))^2$ , QUE ES EL MISMO PARA TODAS LAS SIM)

\subsection{Inciso 6}

En base al ejercicio $5.4)$ de la guía de ejercicios, dada una normal bivariada $(X_1,X_2)$, se puede obtener que:

\[X_1|X_2=x_2 \sim \mathcal{N}(\mu=\mu_{x_1}+\rho \frac{\sigma_{x_1}}{\sigma_{x_2}} (x_2 - \mu_{x_2}) , \sigma^2 = \sigma_{x_1}^2 (1- \rho ^2)) \]

En este caso, se usó $\rho=0.965$  y que $X_1$ y $X_2$ son normales estándar. 

\[\implies X_1|X_2=1 \sim \mathcal{N}(\mu=0.965 , \sigma^2 = 0.068) \]
\[\implies \mathbb{P}(X_1 \leq 1|X_2=1) \approx 0.5531 \]

Dado que la simulación tiene un rango finito de puntos, se buscó aumentar la cantidad de simulaciones,para que al marginar, la probabilidad no varíe  demasiado y se parezca al valor esperado. Por esto mismo, se aumentó la cantidad de número aleatorios (o pseudo-aleatorios) generados de 10 000 a 1 000 000. Además, se reemplazó la condición de $X_2 = 1$ por una condición más abarcativa para que se aproximaran mejor los puntos de la simulación: se utilizó la condición $0.95 \leq X_2 < 1.05$. Los resultados obtenidos son:\\

(PRESENTAR RESULTADOS EN TABLA CON CASOS FAVORABLES, CASOS TOTALES , PROB CALCULADA Y PROB ESPERADA. SI QUERES PODEMOS PONER LO QUE MANDE DE LAS SIMULACIONES DE 10 000 MUESTRAS Y ALGUNA DIFERENCIA O ERROR DE SIMULACION)

\end{document}