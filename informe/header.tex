% header things...

% Lorem Ipsum
\usepackage{lipsum}

% SI units
\usepackage{siunitx}

\usepackage[a4paper,headheight=16pt,scale={0.7,0.8},hoffset=0.5cm]{geometry}


%
% El paquete amsmath agrega algunas funcionalidades extra a las fórmulas. 
% Además defino la numeración de las tablas y figuras al estilo "Figura 2.3", 
% en lugar de "Figura 7". (Por lo tanto, aunque no uses fórmulas, si querés
% este tipo de numeración dejá el paquete amsmath descomentado).
%
\usepackage{amsmath}
\numberwithin{equation}{section}
%\numberwithin{figure}{section}
\numberwithin{table}{section}

%
% Para tener cabecera y pie de página con un estilo personalizado:
%
\usepackage{fancyhdr}

\usepackage[hang,bf]{caption}

\usepackage{circuitikz}

%
%Para que las figuras no floten:
%
\usepackage{float}
\let\origfigure\figure
\let\endorigfigure\endfigure
\renewenvironment{figure}[1][2] {
    \expandafter\origfigure\expandafter[H]
} {
    \endorigfigure
}

% Command Prob.& Stats.
\newcommand{\Expect}{{\rm I\kern-.3em E}}
\newcommand{\Var}{{\rm \kern-.3em Var}}


\usepackage{caption}
\usepackage{subcaption}
\usepackage{listings}

\usepackage{color} %red, green, blue, yellow, cyan, magenta, black, white
\definecolor{mygreen}{RGB}{28,172,0} % color values Red, Green, Blue
\definecolor{mylilas}{RGB}{170,55,241}
\definecolor{mygray}{rgb}{0.5,0.5,0.5}
\definecolor{mymauve}{rgb}{0.58,0,0.82}
\definecolor{myblue}{rgb}{0.33,0.33,0.99}

\lstset{language=python,%    
    basicstyle=\color{red},
    breaklines=true,%
    morekeywords={matlab2tikz},
    keywordstyle=\color{blue},%
    morekeywords=[2]{1}, keywordstyle=[2]{\color{green}},
    identifierstyle=\color{black},%
    stringstyle=\color{mygreen},
    commentstyle=\color{mygray},%
    showstringspaces=false,%without this there will be a symbol in the places where there is a space
    numbers=left,%
    numberstyle={\tiny \color{mygray}},% size of the numbers
    numbersep=3pt, % this defines how far the numbers are from the text
    emph=[1]{for,end,break},emphstyle=[1]\color{blue}, %some words to emphasise
    emph=[2]{word1,word2}, emphstyle=[2]{style},    
}